\documentclass[justified, openany]{tufte-book}

\usepackage[utf8]{inputenc}
\usepackage[english]{babel}
\usepackage{blindtext}
\usepackage{todonotes}

\setcounter{secnumdepth}{1}
\setcounter{tocdepth}{1}

% turn on numbering for parts and chapters

\usepackage{amsthm, amsmath, amssymb}
\usepackage{setspace, graphicx, enumerate}

\usepackage{stmaryrd}

% For nicely typeset tabular material
\usepackage{booktabs}

\usepackage{listings}
\lstloadlanguages{Haskell}

\usepackage{permute}

\newcommand{\monthyear}{%
  \ifcase\month\or January\or February\or March\or April\or May\or June\or
  July\or August\or September\or October\or November\or
  December\fi\space\number\year
}

% For graphics / images
\usepackage{graphicx}
\usepackage{fancyvrb}
\fvset{fontsize=\normalsize}
\usepackage{xspace}

\usepackage{pgf, tikz}
\usetikzlibrary{arrows,automata,decorations.pathmorphing,backgrounds,positioning,fit,petri,shapes.geometric,calc}

\usepackage{units}

\usepackage{bibentry}

\usepackage{tcolorbox}

\usepackage{enumitem}

\usepackage{bm}

\usepackage{mmacells}

\usepackage{oplotsymbl}

\usepackage{pdfpages}

\usepackage[nottoc]{tocbibind}

\theoremstyle{plain}% default 
\newtheorem{thm}{Theorem}[chapter] 
\newtheorem{lem}[thm]{Lemma} 
\newtheorem{prop}[thm]{Proposition} 
\newtheorem*{cor}{Corollary} 

\theoremstyle{definition} 
\newtheorem{defn}[thm]{Definition}
\newtheorem{conj}[thm]{Conjecture}
\newtheorem{exmp}[thm]{Example}
\newtheorem{exer}[thm]{Exercise}

\newtheorem{proofpart}{Proof Part}[thm]

\theoremstyle{remark} 
\newtheorem*{rem}{Remark} 
\newtheorem*{note}{Note} 
\newtheorem{case}{Case}
\newtheorem{thminv}{Invariant}[chapter] 


\newtcolorbox{problem}{title={Problem}}

\newenvironment{dedication}
    {\vspace{6ex}\begin{quotation}\begin{center}\begin{em}\begin{large}}
    {\par\end{large}\end{em}\end{center}\end{quotation}}

\mmaDefineMathReplacement[≤]{<=}{\leq}
\mmaDefineMathReplacement[≥]{>=}{\geq}
\mmaDefineMathReplacement[≠]{!=}{\neq}
\mmaDefineMathReplacement[→]{->}{\to}[2]
\mmaDefineMathReplacement[⧴]{:>}{:\hspace{-.2em}\to}[2]
\mmaDefineMathReplacement{∉}{\notin}
\mmaDefineMathReplacement{∞}{\infty}
\mmaDefineMathReplacement{𝕕}{\mathbbm{d}}


\mmaSet{
  morefv={gobble=2},
  linklocaluri=mma/symbol/definition:#1,
  morecellgraphics={yoffset=1.9ex}
}



\title{Math notes - How many trailing zeros}
\author{Uwe Hoffmann}
\hypersetup{colorlinks, pdftitle={Math notes - How many trailing zeros}}

\begin{document}

\setcounter{chapter}{1}
\section*{How many trailing zeros in $n!$}

\newthought{Greatest dividing exponent} and its properties is the topic of the problem in this note.\index{greatest dividing exponent}

\vspace{10 mm}
\begin{problem}
Write a program that calculates for an arbitrary positive integer $n$ how many trailing zeros there are in $n!$.
\end{problem}

Let's first try to figure out for any natural number $n$ what the number of trailing zeros is. A useful concept here is the greatest dividing exponent \cite{mathworld01}:

\begin{defn}
The greatest dividing exponent  $gde(n, b)$ of a base  $b$ with respect to a number  $n$ is the largest integer value of  $k$ such that $b^k \mid n$, where $b^k \leq n$.
\end{defn}

\begin{lem}\label{mingde}
\begin{equation*}
gde(n, a b) = min(gde(n, a), gde(n, b))  \text{, with } (a, b) = 1
\end{equation*}
\end{lem}

\begin{proof}
Assume $gde(n,a) \leq gde(n, b)$. Then $a^{gde(n, a)} \mid n$ and $b^{gde(n, a)} \mid n$, with $(a^{gde(n, a)}, b^{gde(n, a)}) = 1$, so $(ab)^{gde(n, a)} \mid n$. By definition of $gde$ we then have $gde(n, a) \leq gde(n, ab)$. 

We also have $(ab)^{gde(n, ab)} \mid n$, so $a^{gde(n, ab)} \mid n$. By definition of $gde$ we then have $gde(n, a) \geq gde(n, ab)$. 

It follows that $gde(n, a)= gde(n, ab)$.
\end{proof}

It's clear that the number of trailing zeros of $n$ equals $gde(n, 10)$. From lemma \ref{mingde} we are looking for $min(gde(n!, 2), gde(n!, 5))$.\\

\begin{lem}\label{facgde}
\begin{equation*}
gde(n!, p) = \sum_{k = 1}^{\lfloor log_p n \rfloor} \left\lfloor\frac{n}{p^k}\right\rfloor \text{, for a prime } p \leq n
\end{equation*}
\end{lem}

\begin{proof}
We define the following subsets of $\{1,\ldots,n\}$:

\begin{equation*}
M_p^k = \{i : 1 \leq i \leq n: p^k \mid i\}
\end{equation*}

For $k > \lfloor log_p n \rfloor$ the sets $M_p^k$ are empty, so we only consider $k \leq \lfloor log_p n \rfloor$. Each member of one set $M_p^k$ contributes $k$ to $gde(n!, p)$, so the whole set contributes $k | M_p^k |$.  From $p^k \mid i$ it follows that also $p^{k - 1} \mid i$, so $M_p^k \subseteq M_p^{k - 1}$ for $k = 2,\ldots,\lfloor log_p n \rfloor$. Being careful not to count the contributions more than once we get:
\begin{equation*}
gde(n!, p) = \sum_{k = 1}^{\lfloor log_p n \rfloor}  | M_p^k |
\end{equation*}
With $| M_p^k | = \left\lfloor\frac{n}{p^k}\right\rfloor$ we conclude the proof.
\end{proof}

\begin{lem}\label{gdetwo}
\begin{equation*}
gde(n!, 2) \geq gde(n!, 5) \text{ for any } n \geq 1
\end{equation*}
\end{lem}

\begin{proof}
Plugging in the expression of $gde$ from lemma \ref{facgde} into the claim of this lemma we get:
\begin{equation*}
   gde(n!, 2) \geq gde(n!, 5) \Leftrightarrow  \sum_{k = 1}^{\left\lfloor log_2 n\right\rfloor} \left\lfloor\frac{n}{2^k}\right\rfloor  \geq \sum_{k = 1}^{\left\lfloor log_5 n\right\rfloor} \left\lfloor\frac{n}{5^k}\right\rfloor
\end{equation*}

\noindent We establish:
\begin{equation*}
\begin{split}
   log_2 n  \geq log_5 n & \Leftrightarrow  log_2 n  \geq log_2 n\ log_5 2 \\
   & \Leftrightarrow  1 \geq  log_5 2 \text{, which is true}
\end{split}   
\end{equation*}

\noindent For each $1 \leq k \leq \left\lfloor log_5 n\right\rfloor$ we have:
\begin{equation*}
 \left\lfloor\frac{n}{2^k}\right\rfloor  \geq \left\lfloor\frac{n}{5^k}\right\rfloor
\end{equation*}
and for $\left\lfloor log_5 n\right\rfloor + 1 \leq k \leq \left\lfloor log_2 n\right\rfloor$ we have:
\begin{equation*}
 \left\lfloor\frac{n}{2^k}\right\rfloor > 0
\end{equation*}

\noindent Adding up the inequalities establishes the claim.
\end{proof}

\noindent From the three lemmas we found that:
\begin{equation*}
\begin{split}
\text{(number of trailing zeros in } n! )  & = gde(n!, 10)  \\
&= min(gde(n!, 2), gde(n!, 5))  \\
& = gde(n!, 5)  \\
& = \sum_{k = 1}^{\lfloor log_5 n \rfloor}   \left\lfloor\frac{n}{5^k}\right\rfloor
\end{split}   
\end{equation*}
\noindent so our program needs to calculate the expression:
\begin{equation*}
\sum_{k = 1}^{\lfloor log_5 n \rfloor}   \left\lfloor\frac{n}{5^k}\right\rfloor
\end{equation*}

\noindent The following small Haskell function does it:

\lstinputlisting[language=Haskell, basicstyle=\small, frame=trBL, caption={Haskell code}]{gdefac.hs}

\noindent It works on tuples of numbers. It keeps dividing the second number in the tuple by 5 until zero and adding the division results together into the first number of the tuple. In the end it returns the first number in the tuple.

\bibliographystyle{plainnat}
\bibliography{../common/math}

\end{document}

