\documentclass[justified, openany]{tufte-book}

\usepackage[utf8]{inputenc}
\usepackage[english]{babel}
\usepackage{blindtext}
\usepackage{todonotes}

\setcounter{secnumdepth}{1}
\setcounter{tocdepth}{1}

% turn on numbering for parts and chapters

\usepackage{amsthm, amsmath, amssymb}
\usepackage{setspace, graphicx, enumerate}

\usepackage{stmaryrd}

% For nicely typeset tabular material
\usepackage{booktabs}

\usepackage{listings}
\lstloadlanguages{Haskell}

\usepackage{permute}

\newcommand{\monthyear}{%
  \ifcase\month\or January\or February\or March\or April\or May\or June\or
  July\or August\or September\or October\or November\or
  December\fi\space\number\year
}

% For graphics / images
\usepackage{graphicx}
\usepackage{fancyvrb}
\fvset{fontsize=\normalsize}
\usepackage{xspace}

\usepackage{pgf, tikz}
\usetikzlibrary{arrows,automata,decorations.pathmorphing,backgrounds,positioning,fit,petri,shapes.geometric,calc}

\usepackage{units}

\usepackage{bibentry}

\usepackage{tcolorbox}

\usepackage{enumitem}

\usepackage{bm}

\usepackage{mmacells}

\usepackage{oplotsymbl}

\usepackage{pdfpages}

\usepackage[nottoc]{tocbibind}

\theoremstyle{plain}% default 
\newtheorem{thm}{Theorem}[chapter] 
\newtheorem{lem}[thm]{Lemma} 
\newtheorem{prop}[thm]{Proposition} 
\newtheorem*{cor}{Corollary} 

\theoremstyle{definition} 
\newtheorem{defn}[thm]{Definition}
\newtheorem{conj}[thm]{Conjecture}
\newtheorem{exmp}[thm]{Example}
\newtheorem{exer}[thm]{Exercise}

\newtheorem{proofpart}{Proof Part}[thm]

\theoremstyle{remark} 
\newtheorem*{rem}{Remark} 
\newtheorem*{note}{Note} 
\newtheorem{case}{Case}
\newtheorem{thminv}{Invariant}[chapter] 


\newtcolorbox{problem}{title={Problem}}

\newenvironment{dedication}
    {\vspace{6ex}\begin{quotation}\begin{center}\begin{em}\begin{large}}
    {\par\end{large}\end{em}\end{center}\end{quotation}}

\mmaDefineMathReplacement[≤]{<=}{\leq}
\mmaDefineMathReplacement[≥]{>=}{\geq}
\mmaDefineMathReplacement[≠]{!=}{\neq}
\mmaDefineMathReplacement[→]{->}{\to}[2]
\mmaDefineMathReplacement[⧴]{:>}{:\hspace{-.2em}\to}[2]
\mmaDefineMathReplacement{∉}{\notin}
\mmaDefineMathReplacement{∞}{\infty}
\mmaDefineMathReplacement{𝕕}{\mathbbm{d}}


\mmaSet{
  morefv={gobble=2},
  linklocaluri=mma/symbol/definition:#1,
  morecellgraphics={yoffset=1.9ex}
}



\title{Math notes - Points on circle}
\author{Uwe Hoffmann}
\hypersetup{colorlinks, pdftitle={Math notes - Points on circle}}

\begin{document}

\setcounter{chapter}{1}
\section*{Points on circle}

\vspace{10 mm}
\begin{problem}
$N$ distinct points, numbered from $0$ onwards, are located on a circle (in the rest of this problem all point numbers are taken $\mathbf{mod} N$). Point $i + 1$ is the clockwise neighbor of point $i$. An integer array, $dist[0 \ldots N)$, is given such that $dist.i$ is the distance (along the circle) between points $i$ and  $i + 1$. Derive a program to determine whether four of these points form a rectangle.
\end{problem}

We adopt the same notation used in \textit{Programming in the 1990s} \cite{Cohen90} and \textit{Programming, The Derivation of Algorithms}\cite{Kaldewaij90}: The notation of function application is the "dot" notation with name of function, followed by arguments, each separated by a dot. The notation of quantified expressions has the operator followed by the bounded variables, then a colon followed by the range for the bounded variables and ended with a colon and the actual expression. So

\begin{equation*}	 
	(\sum k : i \leq k < j : x_k)
\end{equation*}

\noindent corresponds to the more classical mathematical notation $\sum_{k = i}^{ j - 1}x_k$. 

\noindent For our derivation steps in predicate calculus we will use the following notation:

\begin{equation*}
\begin{array}{lcl}
		&&A \\
	      &=& \{  \mbox{reason why A equals B} \} \\      
                  &&B \\
                &\leq& \{ \mbox{reason why B is less than C} \} \\
                  && C  
   \end{array}
\end{equation*}

\noindent We are asked to solve $S$ in \medskip

$\|[$

\verb|    | \textbf{con} $N:\ int;\ \{N \geq 4\}$

\verb|        | $dist(i: 0 \leq i < N):\ int;\ \{\forall i: 0 \leq i < N: dist.i > 0\}$

\verb|    | \textbf{var} $r: bool$;

\verb|        | S

\verb|    | $\{r: r \equiv (\exists\ 4\  \text{points that form a rectangle}) \}$

$]\|$

\medskip

\noindent Let's first develop a more manageable postcondition. Evidently four points that form a rectangle is equivalent to two pairs of diametral opposing points. We introduce a function for the set of all indices from point $x$ to point $y$ in clockwise direction along the circle:

\begin{equation*}
\begin{array}{ll}
& I: [0, \ldots, N) \rightarrow [0, \ldots, N) \rightarrow 2^{[0, \ldots, N)} \\
& I.x.y :=
\begin{cases}
[x, \ldots, y)\qquad,\  x \leq y \\
[x, \ldots, N) \bigcup\  [0, \ldots, y) \quad,\ x > y
 \end{cases}
 \end{array}
\end{equation*}

\noindent Let $C$ be the circumference of the circle.
\noindent We define function
\begin{equation*}
\begin{array}{ll}
   & f : [0, \ldots, N) \rightarrow [0, \ldots, N) \rightarrow int \\
   & f.x.y := C - 2 (\sum i: i \in I.x.y : dist.i)
\end{array}
\end{equation*}

\noindent We want to find the number of diametral opposing pairs of points:\medskip

$\|[$

\verb|    | \textbf{con} $N:\ int;\ \{N \geq 2\}$

\verb|        | $dist(i: 0 \leq i < N):\ int;\ \{\forall i: 0 \leq i < N: dist.i > 0\}$

\verb|    | \textbf{var} $r: int$;

\verb|        | S

\verb|    | $\{r: r = (\#\ x, y: 0 \leq x < N,\  0 \leq y < N: f.x.y = 0) \}$

$]\|$

\medskip

\begin{lem}\label{slope}
The function $f$ is increasing in its first argument and decreasing in its second argument.
\end{lem}

\begin{proof}
$f$ is increasing in its first argument:
\begin{equation*}
\begin{array}{lcl}
		&& f.(x + 1).y \\
	      &=& \{ \mbox{definition of $f$} \} \\      
                  && C - 2 (\sum i: i \in I.(x + 1).y : dist.i) \\
                &=& \{  I.(x + 1).y = I.x.y \setminus \{x\} \} \\
                  &&  C - 2((\sum i: i \in I.x.y : dist.i) - dist.x)  \\
                 &=& \{ \mbox{definition of $f$} \} \\
                 && f.x.y + 2 dist.x \\
                 &>& \{ dist.x > 0 \} \\
                && f.x.y
   \end{array}
\end{equation*}

$f$ is decreasing in its second argument:
\begin{equation*}
\begin{array}{lcl}
		&& f.x.(y + 1) \\
	      &=& \{ \mbox{definition of $f$} \} \\      
                  && C - 2 (\sum i: i \in I.x .(y + 1) : dist.i) \\
                &=& \{  I.x.(y + 1) = I.x.y\ \bigcup\  \{y\} \} \\
                  &&  C - 2((\sum i: i \in I.x.y : dist.i) + dist.y)  \\
                 &=& \{ \mbox{definition of $f$} \} \\
                 && f.x.y - 2 dist.y \\
                 &<& \{ dist.y > 0 \} \\
                && f.x.y
   \end{array}
\end{equation*}
\end{proof}

\noindent Looking at the postcondition 
\begin{equation*}
\{r: r = (\#\ x, y: 0 \leq x < N,\  0 \leq y < N: f.x.y = 0) \}
\end{equation*}
 we define the function
 \begin{equation*}
 G.a.b = (\#\ x, y: a \leq x < N,\  b \leq y < N: f.x.y = 0)
\end{equation*}
and we will maintain the invariants:
\begin{equation*}
\begin{array}{lcl}
     P_0 &:& G.0.0 = r + G.a.b \\      
     P_1 &:& 0 \leq a \leq N\\
     P_2 &:& 0 \leq b \leq N
   \end{array}
\end{equation*}

\noindent The initial values $r, a, b := 0, 0, 0$ satisfy  the invariants and
\begin{equation*}
a = N \vee b = N \Rightarrow G.a.b = 0 \Rightarrow r = G.0.0
\end{equation*}
establishes the postcondition, so we can stop when $a = N \vee b = N$.

\noindent So far we have\medskip

$\|[$

\verb|    | \textbf{con} $N:\ int;\ \{N \geq 4\}$

\verb|        | $dist(i: 0 \leq i < N):\ int;\ \{\forall i: 0 \leq i < N: dist.i > 0\}$

\verb|    | \textbf{var} $a, b, r: int$;

\verb|    | $a, b, r := 0, 0, 0$;

\verb|    | \textbf{do} $a \neq N \wedge b \neq N$

\verb|       | S

\verb|    | \textbf{od}

\verb|    | $\{r: r = G.0.0 \}$

$]\|$

\medskip

\noindent We need to increment $a, b$ and maintain the invariants:
\begin{equation*}
\begin{array}{lcl}
		&& G.a.b \\
	      &=& \{ \mbox{definition of $G$} \} \\      
                  && (\#\ x, y: a \leq x < N,\  b \leq y < N: f.x.y = 0) \\
                &=& \{  \text{range split } x = a \} \\
                  &&  G.(a + 1).b + (\# y: b \leq y < N:f.a.y = 0)  \\
                 &=& \{ \mbox{$f$ is decreasing in second argument (\ref{slope}), and assume } f.a.b < 0 \} \\
                 && G.(a + 1).b 
   \end{array}
\end{equation*}
so $f.a.b < 0 \Rightarrow G.a.b = G.(a + 1).b$. Similarly
\begin{equation*}
\begin{array}{lcl}
		&& G.a.b \\
	      &=& \{ \mbox{definition of $G$} \} \\      
                  && (\#\ x, y: a \leq x < N,\  b \leq y < N: f.x.y = 0) \\
                &=& \{  \text{range split } y = b \} \\
                  &&  G.a.(b + 1) + (\# x: a \leq y < N:f.x.b = 0)  \\
                 &=& \{ \mbox{$f$ is increasing in second argument (\ref{slope}), and assume } f.a.b > 0 \} \\
                 && G.a.(b + 1) 
   \end{array}
\end{equation*}
so $f.a.b > 0 \Rightarrow G.a.b = G.a.(b + 1)$. Also for the case $f.a.b = 0$ we have
\begin{equation*}
\begin{array}{lcl}
		&& r + G.a.b \\
	      &=& \{ \mbox{definition of $G$} \} \\      
                  && r + (\#\ x, y: a \leq x < N,\  b \leq y < N: f.x.y = 0) \\
                &=& \{  \text{range split } x = a \} \\
                  &&  r + G.(a + 1).b + (\# y: b \leq y < N:f.a.y = 0)  \\
                 &=& \{ \mbox{$f$ is decreasing in second argument (\ref{slope}), and assume } f.a.b = 0 \} \\
                 && (r + 1) +  G.(a + 1).b 
 \end{array}
\end{equation*} 

Our program becomes\medskip

$\|[$

\verb|    | \textbf{con} $N:\ int;\ \{N \geq 4\}$

\verb|        | $dist(i: 0 \leq i < N):\ int;\ \{\forall i: 0 \leq i < N: dist.i > 0\}$

\verb|    | \textbf{var} $a, b, r: int$;

\verb|    | $a, b, r := 0, 0, 0$;

\verb|    | \textbf{do} $a \neq N \wedge b \neq N$

\verb|       | \textbf{if}

\verb|       | $\square\  f.a.b > 0 \rightarrow b := b + 1$

\verb|       |  $\square\  f.a.b < 0 \rightarrow a := a + 1$ 

\verb|       |  $\square\  f.a.b = 0 \rightarrow a, r := a + 1, r + 1$ 

\verb|       | \textbf{fi}

\verb|    | \textbf{od}

\verb|    | $\{r: r = G.0.0 \}$

$]\|$

\medskip

\noindent We cannot have $f$ in the program text so the last thing we have to do is eliminate $f$. We do this by introducing a new variable $c: int$ and maintaining the additional invariant $P_3: c = f.a.b$. Lemma \ref{slope} already showed us the expressions for $f$ when the first or the second argument increase, so our final program looks like this\footnote{The program is bound by the function $2 N - a - b$ so it is $O(N)$. The solution is an example of the slope search technique.}\medskip

$\|[$

\verb|    | \textbf{con} $N:\ int;\ \{N \geq 4\}$

\verb|        | $dist(i: 0 \leq i < N):\ int;\ \{\forall i: 0 \leq i < N: dist.i > 0\}$

\verb|    | \textbf{var} $a, b, c, r: int$;

\verb|    | $a, b, c, r := 0, 0, C, 0$;

\verb|    | \textbf{do} $a \neq N \wedge b \neq N$

\verb|       | \textbf{if}

\verb|       | $\square\  c > 0 \rightarrow b, c := b + 1, c - 2 dist.b$

\verb|       |  $\square\  c < 0 \rightarrow a, c := a + 1, c + 2 dist.a$ 

\verb|       |  $\square\  c = 0 \rightarrow a, c, r := a + 1, 2 dist.a, r + 1$ 

\verb|       | \textbf{fi}

\verb|    | \textbf{od}

\verb|    | $\{r: r = G.0.0 \}$

$]\|$

\bibliographystyle{plainnat}
\bibliography{../common/math}

\end{document}

