\newthought{Select exercises on sequences and series} from Chapter 3 of the \textit{Lectures on Real Analysis} textbook\footnote{\bibentry{lárusson2012lectures}}.

\begin{tcolorbox}[title={Exercise 3.17, page 35}]
(a) Let $a \geq 0$ and $n \in \mathbb{N}$, $n \geq 2$. Show that 

$$
(1 + a)^n \geq \frac{1}{2} n (n -1)a^2
$$

(b) Show that $n^{\frac{1}{n}} \to 1$ as $n \to \infty$.
\end{tcolorbox}

\begin{solution}
(a) Using the binomial expansion, we get

$$
(1 + a)^n = \sum_{k = 0}^n \binom{n}{k} a^k = 1 + n a + \frac{1}{2} n (n -1)a^2 + \ldots \geq \frac{1}{2} n (n -1)a^2
$$

(b) Using the inequality from (a) with $a = n^{\frac{1}{n}} - 1$ we get

$$
n = (n^{\frac{1}{n}} - 1 + 1)^n \geq \frac{1}{2} n (n -1) (n^{\frac{1}{n}} - 1)
$$

So $\frac{2}{n-1} \geq (n^{\frac{1}{n}} - 1)$ and $n^{\frac{1}{n}} \to 1$.
\end{solution}

\begin{tcolorbox}[title={Exercise 3.18, page 35}]
Consider the recursively defined sequence $(a_n)$ with $a_1 = 3$ and $a_{n+1}=\frac{a_n}{2} + \frac{3}{a_n}$. Show that $(a_n)$ converges and find its limit.
\end{tcolorbox}

\begin{solution}
Let's first prove by induction that $\forall n \in \mathbb{N}: 2 < a_n \leq 3$:

It's true for $a_1=3$. Assume it is true for a given $n$ and let's do the induction step.

$$
a_{n+1} = \frac{a_n}{2} + \frac{3}{a_n} > \frac{2}{2} + \frac{3}{3} = 2
$$

Also

$$
a_{n+1} = \frac{a_n}{2} + \frac{3}{a_n} \leq \frac{3}{2} + \frac{3}{2} = 3
$$

At least we know $(a_n)$ is bounded. Let us spy a little and assume $(a_n)$ does converge, say to limit $L$. Then $L$ must satisfy: 

$$
L = \frac{L}{2} + \frac{3}{L}
$$

which works out to $L = \sqrt{6}$.

Let's try with a simpler sequence $(b_n)$ such that $a_n = b_n \sqrt{6}$. 

\begin{align*}
  a_{n+1} = b_{n + 1} \sqrt{6} &= \frac{a_n}{2} + \frac{3}{a_n} \\
                              &= \frac{b_n \sqrt{6}}{2} + \frac{3}{b_n \sqrt{6}} \\
                              &= \frac{b_n \sqrt{6}}{2} + \frac{\sqrt{6}}{2 b_n}
\end{align*}

So $(b_n)$ satisfies $b_{n+1} = \frac{1}{2}(b_n + \frac{1}{b_n})$. We prove that $(b_n)$ is monoton decreasing:

\begin{align*}
  b_{n+1} \leq b_n & \Leftrightarrow \\
  \frac{1}{2}(b_n + \frac{1}{n}) \leq b_n & \Leftrightarrow \\
  b_n^2 + 1 \leq 2 b_n^2  & \Leftrightarrow \\
  b_n^2 \geq 1 & \Leftrightarrow \\
  b_n \geq 1
\end{align*}

We use the AGM inequality\footnote{For positive $x$ and $y$ we have $(\sqrt{x} + \sqrt{y})^2 \geq 0$ which when expanded ends up at $\frac{x + y}{2} \geq \sqrt{xy}$. } and show:

$$
b_{n+1} = \frac{1}{2}(b_n + \frac{1}{b_n}) \geq \sqrt{b_n \frac{1}{b_n}} = 1
$$

So $(b_n)$ is monoton decreasing and bounded below by $1$, so $(b_n)$ converges, and so does $(a_n)$: $b_n \to 1$ and $a_n \to \sqrt{6}$.

\end{solution}

\begin{tcolorbox}[title={Exercise 3.23, page 36}]
Let $\sum a_n$ be a series. Set $a^+_n = max\{0, a_n\}$ and $a^-_n = min\{0, a_n\}$. Consider the series $\sum a^+_n$ and $\sum a^-_n$.

(a) Prove that $\sum a_n$ is absolutely convergent if and only if $\sum a^+_n$ and $\sum a^-_n$ both converge. Then $\sum a_n = \sum a^+_n + \sum a^-_n$.

(b) Prove that if $\sum a_n$ is conditionally convergent, then $\sum a^+_n$ and $\sum a^-_n$ both diverge. 
\end{tcolorbox}

\begin{solution}
We will use the partial sums:

\begin{align*}
s_n &= \sum_{k = 1}^n a_k, \quad s^a_n = \sum_{k = 1}^n |a_k| \\
s^+_n &= \sum_{k = 1}^n a^+_k, \quad s^-_n = \sum_{k = 1}^n a^-_k
\end{align*}

(a) $(\Rightarrow)$ 

We have $\forall n \in \mathbb{N}: |a_n| \geq a^+_n \text{ and } |a_n| \geq (-1) a^-_n$. Using the comparison test we find $\sum a^+_n$ and $\sum a^-_n$ converge.

$(\Leftarrow)$ $\sum a^+_n$ and $\sum a^-_n$ converge, so then also $\sum a^+_n + (-1) \sum a^-_n$ converges. But $s^a_n = s^+_n + (-1)s^-_n$, so $\sum |a_n|$ converges too.

(b) $\sum a_n$ converges conditionally. If both $\sum a^+_n$ and $\sum a^-_n$ converge, then from (a) we would have $\sum a_n$ converges absolutely, contradicting the premise. So at least one of $\sum a^+_n$ or $\sum a^-_n$ must diverge. 

Assume $\sum a^+_n$ diverges (the other case is similar). $s^+_n$ is monotonically increasing and divergent, so it is unbounded. We have $s^+_n = s_n - s^-_n$ and $s_n$ is bounded. It follows that $s^-_n$ has to be unbounded, so $\sum a^-_n$ diverges also.

\end{solution}

\begin{tcolorbox}[title={Exercise 3.24, page 36}]
Let $\sum a_n$ be a conditionally convergent series. Prove that for every $\sigma \in \mathbb{R}$ there is a rearrangement of $\sum a_n$ that converges to $\sigma$.
\end{tcolorbox}

\begin{solution}
We will construct this rearrangement.

We know from the previous exercise that both $\sum a^+_n$ and $\sum a^-_n$ diverge and both $s^+_n$ and $s^-_n$ are unbounded. 

Assume first that $\sigma > 0$ (the other case is similar). Since $s^+_n$ is unbounded, there exists\footnote{This $N_1$ has to exist because $s^+_n$ is unbounded. If it was only zeros it would converge and be bounded.} a $N_1 \in \mathbb{N}$ such that 

\begin{align*}
\sum_{k = 1}^{N_1-1} a^+_k  &\leq \sigma \\
\sum_{k = 1}^{N_1} a^+_k &> \sigma
\end{align*}

Let $d_1 = |\sum_{k = 1}^{N_1} a^+_k - \sigma|$. We see that $0 < d_1 \leq |a^+_{N_1}|$. Our rearrangement will start with the first $N_1$ terms from $\sum a^+_n$. For the next terms we turn to $\sum a^-_n$. $s^-_n$ is also unbounded, so there exists a $M_1 \in \mathbb{N}$ such that

\begin{align*}
\sum_{k = 1}^{M_1-1} a^-_k &\geq d_1 \\
\sum_{k = 1}^{M_1} a^-_k &< d_1
\end{align*}

We add the first $M_1$ terms from $\sum a^-_n$ to the rearrangement. Let $d_2 = |\sum_{k = 1}^{N_1} a^+_k + \sum_{k = 1}^{M_1} a^-_k - \sigma|$. We see that $0 < d_2 \leq |a^-_{M_1}|$.

Next we go back to $\sum a^+_n$ for more terms. The tail of $\sum a^+_n$ starting at $N_1 + 1$ is also unbounded, so there must exist a $N_2$ such that

\begin{align*}
\sum_{k = N_1 + 1}^{N_2-1} a^+_k  &\leq d_2 \\
\sum_{k = N_1 + 1}^{N_2} a^+_k &> d_2
\end{align*}

We add the terms $\sum_{k = N_1 + 1}^{N_2} a^+_k$ to the rearrangement and define 

$$
d_3 = |\sum_{k = 1}^{N_1} a^+_k + \sum_{k = 1}^{M_1} a^-_k + \sum_{k = N_1 + 1}^{N_2} a^+_k - \sigma|
$$

We see that $0 < d_3 \leq |a^+_{N_2}|$.

We go back down with the help of terms from the tail of $\sum a^-_n$ starting at $M_1$, a tail that is also unbounded. There must exist a $M_2$ such that

\begin{align*}
\sum_{k = M_1 + 1}^{M_2-1} a^+_k  &\geq d_3 \\
\sum_{k = M_1 + 1}^{M_2} a^+_k & < d_3
\end{align*}

We add the terms $\sum_{k = M_1 + 1}^{M_2} a^-_k$ to the rearrangement and define 

$$
d_4 = |\sum_{k = 1}^{N_1} a^+_k + \sum_{k = 1}^{M_1} a^-_k + \sum_{k = N_1 + 1}^{N_2} a^+_k + \sum_{k = M_1 + 1}^{M_2} a^-_k - \sigma|
$$

We see that $0 < d_4 \leq |a^-_{M_2}|$.

We continue in this way, switching between terms in $\sum a^+_n$ and $\sum a^-_n$, constructing a rearrangement of $\sum a_n$ that has partial sums that have distance $d_n$ from $\sigma$.

The sequence $(d_n)$ of distances is bounded by $(|a_n|)$ and $\sum a_n$ is a conditionally convergent series, so $a_n \to 0$. That means that $d_n \to 0$ and the rearrangement converges to $\sigma$.

\end{solution}
