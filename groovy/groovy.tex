\documentclass[justified, openany]{tufte-book}

\usepackage[utf8]{inputenc}
\usepackage[english]{babel}
\usepackage{blindtext}
\usepackage{todonotes}

\setcounter{secnumdepth}{1}
\setcounter{tocdepth}{1}

% turn on numbering for parts and chapters

\usepackage{amsthm, amsmath, amssymb}
\usepackage{setspace, graphicx, enumerate}

\usepackage{stmaryrd}

% For nicely typeset tabular material
\usepackage{booktabs}

\usepackage{listings}
\lstloadlanguages{Haskell}

\usepackage{permute}

\newcommand{\monthyear}{%
  \ifcase\month\or January\or February\or March\or April\or May\or June\or
  July\or August\or September\or October\or November\or
  December\fi\space\number\year
}

% For graphics / images
\usepackage{graphicx}
\usepackage{fancyvrb}
\fvset{fontsize=\normalsize}
\usepackage{xspace}

\usepackage{pgf, tikz}
\usetikzlibrary{arrows,automata,decorations.pathmorphing,backgrounds,positioning,fit,petri,shapes.geometric,calc}

\usepackage{units}

\usepackage{bibentry}

\usepackage{tcolorbox}

\usepackage{enumitem}

\usepackage{bm}

\usepackage{mmacells}

\usepackage{pdfpages}

\usepackage[nottoc]{tocbibind}

\theoremstyle{plain}% default 
\newtheorem{thm}{Theorem}[chapter] 
\newtheorem{lem}[thm]{Lemma} 
\newtheorem{prop}[thm]{Proposition} 
\newtheorem*{cor}{Corollary} 

\theoremstyle{definition} 
\newtheorem{defn}[thm]{Definition}
\newtheorem{conj}[thm]{Conjecture}
\newtheorem{exmp}[thm]{Example}
\newtheorem{exer}[thm]{Exercise}

\newtheorem{proofpart}{Proof Part}[thm]

\theoremstyle{remark} 
\newtheorem*{rem}{Remark} 
\newtheorem*{note}{Note} 
\newtheorem{case}{Case}

\newtcolorbox{problem}{title={Problem}}

\newenvironment{dedication}
    {\vspace{6ex}\begin{quotation}\begin{center}\begin{em}\begin{large}}
    {\par\end{large}\end{em}\end{center}\end{quotation}}



\title{Math notes - Groovy Numbers}
\author{Uwe Hoffmann}
\hypersetup{colorlinks, pdftitle={Math notes - Groovy Numbers}}

\begin{document}

\setcounter{chapter}{1}
\section*{Groovy Numbers}

\vspace{10 mm}
\begin{problem}
$x \in \mathbb{R}$ is said to be a groovy number iff $\exists\  n \in \mathbb{N}$ such that $x = \sqrt{n} + \sqrt{n + 1}$. Prove that if $x$ is groovy, then $\forall r \in \mathbb{N}:\ x^r$ is groovy.
\end{problem}

\subsection{Binomial Expansion}

In this section we explore a property of the binomial power expansion

\begin{equation*}
(a + b)^r = \sum_{k = 0}^r \binom{r}{k} a^{r - k} b^k
\end{equation*}

\noindent We define $\mathbb{N}_r = \{k \in \mathbb{N}_0: 0 \leq k \leq r\}$ and its partition into two subsets \mbox{$\mathbb{N}_r = \mathbb{E}_r \cup \mathbb{O}_r$}, with  \mbox{$\mathbb{E}_r = \{k \in \mathbb{N}_r: k = 2 u, u \in \mathbb{N}_0 \}$} and \\
\mbox{$\mathbb{O}_r = \{k \in \mathbb{N}_r: k  = 2 u + 1, u \in \mathbb{N}_0 \}$}. We then partition the binomial power expansion into two sums:

\begin{equation*}
(a + b)^r = \sum_{k = 0}^r \binom{r}{k} a^{r - k} b^k = \sum_{k \in \mathbb{E}_r}  \binom{r}{k} a^{r - k} b^k + \sum_{k \in \mathbb{O}_r}  \binom{r}{k} a^{r - k} b^k
\end{equation*}

\noindent Let 

\begin{equation*}
E(a, b, r) = \sum_{k \in \mathbb{E}_r}  \binom{r}{k} a^{r - k} b^k \ \text{and} \ O(a, b, r) = \sum_{k \in \mathbb{O}_r}  \binom{r}{k} a^{r - k} b^k
\end{equation*}
  
\noindent Then

\begin{align*}
(a^2 - b^2)^r &= (a + b)^r (a - b)^r \\
&= (E(a, b, r) + O(a, b, r)) (E(a, -b, r) + O(a, -b, r))
\end{align*}

\noindent But

\begin{equation*}
E(a, -b, r) = E(a, b, r) \ \text{and}\  O(a, -b, r) = - O(a, b, r) 
\end{equation*}

\noindent so

\begin{align*}
(a^2 - b^2)^r &= (a + b)^r (a - b)^r \\
&= (E(a, b, r) + O(a, b, r)) (E(a, -b, r) + O(a, -b, r)) \\
&= (E(a, b, r) + O(a, b, r)) (E(a, b, r) - O(a, b, r)) \\
&= E(a, b, r)^2 - O(a, b, r)^2
\end{align*}

\noindent We therefore proved 

\begin{lem}\label{even_odd}
\begin{equation*}
(a^2 - b^2)^r = E(a, b, r)^2 - O(a, b, r)^2
\end{equation*}  
\end{lem}

\subsection{Solution}

Using lemma \ref{even_odd} with $a = \sqrt{n}$ and $b = \sqrt{n + 1}$, we get

\begin{equation}
(-1)^r = E(\sqrt{n}, \sqrt{n + 1}, r)^2 - O(\sqrt{n}, \sqrt{n + 1}, r)^2\tag{L}
\end{equation} 

\begin{lem}\label{squared}
\begin{gather*}
E(\sqrt{n}, \sqrt{n + 1}, r)^2 \in \mathbb{N}, \\
O(\sqrt{n}, \sqrt{n + 1}, r)^2 \in \mathbb{N}
\end{gather*}
\end{lem}
 
\begin{proof}

\noindent We will look at two cases: $r$ even and $r$ odd.

\noindent \textbf{Case 1.} For $r = 2 u$ even we have

\begin{align*}
E(\sqrt{n}, \sqrt{n + 1}, 2 u) &=  \sum_{k = 0}^u  \binom{2 u}{2 k} (\sqrt{n})^{2 u - 2 k} (\sqrt{n + 1})^{2 k} \\
&=  \sum_{k = 0}^u  \binom{2 u}{2 k} (\sqrt{n})^{2 (u - k)} (\sqrt{n + 1})^{2 k} \\
&=  \sum_{k = 0}^u  \binom{2 u}{2 k} n^{u - k} (n + 1)^k \\
\end{align*}

\noindent so $E(\sqrt{n}, \sqrt{n + 1}, r) \in \mathbb{N}$, and therefore $E(\sqrt{n}, \sqrt{n + 1}, r)^2 \in \mathbb{N}$.

\begin{align*}
O(\sqrt{n}, \sqrt{n + 1}, 2 u) &=  \sum_{k = 0}^{u - 1}  \binom{2 u}{2 k + 1} (\sqrt{n})^{2 u - 2 k - 1} (\sqrt{n + 1})^{2 k + 1} \\
&=  \frac{\sqrt{n + 1}}{\sqrt{n}} \sum_{k = 0}^{u - 1}  \binom{2 u}{2 k + 1} (\sqrt{n})^{2 (u - k)} (\sqrt{n + 1})^{2 k} \\
&=  \frac{\sqrt{n + 1}}{\sqrt{n}} \sum_{k = 0}^{u - 1}  \binom{2 u}{2 k + 1} n^{2 (u - k)} (n + 1)^{k} \\
&=  \sqrt{n (n + 1)} \sum_{k = 0}^{u - 1}  \binom{2 u}{2 k + 1} n^{2 (u - k) - 1} (n + 1)^{k} \\
\end{align*}

\noindent so $O(\sqrt{n}, \sqrt{n + 1}, r)^2 \in \mathbb{N}$.

\noindent \textbf{Case 2.}  $r = 2 u + 1$ is handled in a similar fashion by factoring out $\sqrt{n}$ and $\sqrt{n + 1}$ with the remainder $\in \mathbb{N}$.

\end{proof}

\noindent From lemma \ref{squared} and equation (L) it follows that  $E(\sqrt{n}, \sqrt{n + 1}, r)^2$ and $O(\sqrt{n}, \sqrt{n + 1}, r)^2$ are consecutive natural numbers. Let

\begin{equation*}
m = min(E(\sqrt{n}, \sqrt{n + 1}, r)^2, O(\sqrt{n}, \sqrt{n + 1}, r)^2) \in \mathbb{N}
\end{equation*}

\noindent Then

\begin{equation*}
x^r = (\sqrt{n} + \sqrt{n + 1})^r = \sqrt{m} + \sqrt{m + 1}
\end{equation*}

\end{document}

