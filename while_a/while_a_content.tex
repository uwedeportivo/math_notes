\newthought{Loop invariants} is the topic of the problem \footnote{Problem 4 on page 9 from \bibentry{engel2013problem}} in this note.\index{loop invariants}


\vspace{10 mm}
\begin{problem}
We start with the state $(a, b)$ where $a$, $b$ are positive integers. To this initial state we apply the following algorithm:

\begin{lstlisting}[language=Python, basicstyle=\small, label={lst:while_loop}, frame=trBL]
while a > 0:
   if a < b:
      (a,b) = (2a, b - a)
   else:
      (a,b) = (a - b, 2b) 
\end{lstlisting}

For which starting positions does the algorithm stop? In how many steps does it stop,
if it stops? What can you tell about periods and tails?

\end{problem}

We start with $a > 0$ and $b > 0$. We adopt the following notation: $a_i$, $b_i$ are the values after $i \in \mathbb{N}_{\geq 0}$ times through the loop. Before the first time through the loop $a_0 = a$, $b_0 = b$. Let $n = a + b$.

Let's collect some invariants. We will prove all of them by induction on $i \in \mathbb{N}_{\geq 0}$.

\begin{thminv}\label{p1}
$$
  \forall i \geq 0: a_i + b_i = n
$$
\end{thminv}

\begin{proof}
Base case $a_0 + b_0 = a + b = n$ holds by definition of $n$ and $(a_0, b_0)$.
Assume $a_i + b_i = n$. For $a_{i+1} + b_{i+1}$ we have two cases:

Case $a_i < b_i$: Here we have $a_{i+1} = 2 a_i$ and $b_{i+1} = b_i - a_i$. So

$$
a_{i+1} + b_{i+1} = 2 a_i + b_i - a_i = a_i + b_i = n
$$

Case $a_i \geq b_i$: In this case we have  $a_{i+1} = a_i - b_i$ and $b_{i+1} = 2 b_i$. It follows

$$
a_{i+1} + b_{i+1} = a_i - b_i + 2 b_i = a_i + b_i = n
$$

\end{proof}

\begin{thminv}\label{p2}
$$
  \forall i \geq 0: b_i > 0
$$
\end{thminv}

\begin{proof}
This follows almost immediately from definitions \footnote{
Base case $b_0 = b > 0$ holds by definition of $b$. Assume $b_i > 0$. Again we have two cases. If $a_i < b_i$ then $b_{i+1} = b_i - a_i > 0$. 
If $a_i \geq b_i$ then $b_{i+1} = 2 b_i > 0$.
}.

\end{proof}

\begin{thminv}\label{p3}
$$
  \forall i \geq 0: a_i \geq 0
$$
\end{thminv}

\begin{proof}
This also follows from definitions \footnote{
Base case $a_0 = a > 0$ holds by definition of $a$. Assume $a_i \geq 0$. Again we have two cases. If $a_i < b_i$ then $a_{i+1} = 2 a_i \geq 0$. 
If $a_i \geq b_i$ then $a_{i+1} = a_i - b_i \geq 0$.
}.

\end{proof}


\begin{thminv}\label{p4}
$$
  \forall i \geq 0: a_i \equiv 2^i a \mod n
$$
\end{thminv}

\begin{proof}
Base case $a_0 = a = 2^0 a$ trivially holds.
Assume $a_i \equiv 2^i a \mod n$. For $a_{i+1}$ we have two cases:

Case $a_i < b_i$: Here we have $a_{i+1} = 2 a_i$. So

\begin{align*}
  a_{i+1} &= 2 a_i \\
  &\equiv 2 \cdot 2^i a \mod n \\
  &\equiv 2^{i+1} a \mod n
\end{align*}

Case $a_i \geq b_i$: In this case we have  $a_{i+1} = a_i - b_i$. It follows

\begin{align*}
  a_{i+1} &= a_i - b_i \\
  &\equiv a_i + n - b_i \mod n \\
  &\equiv a_i + a_i + b_i - b_i \mod n \\
  &\equiv 2 a_i \mod n \\
  &\equiv 2 \cdot 2^i a \mod n \\
  &\equiv 2^{i+1} a \mod n  
\end{align*}

\end{proof}

We will use these 4 invariants ($a_i \geq 0$, $b_i > 0$, $a_i + b_i = n$ and $a_i \equiv 2^i a \mod n$) to determine for which initial values $a$ and $b$ the loop terminates. To do so we consider $\frac{a}{n}$. Because $0 < a < n$ we know that $\frac{a}{n} \in (0, 1)$. We look at the expansion of $\frac{a}{n}$ in base $2$.

\begin{thm}
If the expansion of $\frac{a}{n}$ is finite with $k$ digits $d_i \in \{0, 1\}$

$$
  \frac{a}{n} = \sum_{i = 1}^k d_i 2^{-i}
$$

then $a_k = 0$ and the loop terminates after $k$ steps.
\end{thm}

\begin{proof}

From 

$$
  \frac{a}{n} = \sum_{i = 1}^k d_i 2^{-i}
$$

we get by multiplying both sides with $2^k n$:

$$
  2^k a = \sum_{i = 1}^k n d_i 2^{k-i} \equiv 0 \mod n
$$

Together with invariant \ref{p4} we get

$$
  a_k \equiv 2^k a \equiv 0 \mod n
$$

and because $a_k \geq 0$, $b_k > 0$, $a_k + b_k = n$ we know that $0 \leq a_k < n$, so it must be that $a_k=0$ and the loop terminates after at most $k$ steps. To show that the loop terminates after exactly $k$ steps, we need to show that $a_j > 0$ for $0 \leq j < k$. We will do this by finding a contradiction. Assume there exists a $j < k$ such that $a_j = 0$. Then it also holds that $2^j a \equiv 0 \mod n$.

From 

$$
  \frac{a}{n} = \sum_{i = 1}^k d_i 2^{-i}
$$

we get by multiplying both sides with $2^j n$:

$$
  2^j a = \sum_{i = 1}^k n d_i 2^{j-i} = \sum_{i = 1}^j n d_i 2^{j-i} + \sum_{i = j+1}^k n d_i 2^{j-i} \equiv 0 \mod n
$$

$2^j a \equiv 0 \mod n$, so $2^j a = n q$ for some $q \in \mathbb{Z}$. Then

$$
  q = \sum_{i = 1}^j d_i 2^{j-i} + \sum_{i = j+1}^k d_i 2^{j-i}
$$

We have $q \in \mathbb{Z}$, $\sum_{i = 1}^j d_i 2^{j-i} \in \mathbb{Z}$, but $\sum_{i = j+1}^k d_i 2^{j-i} \notin \mathbb{Z}$, because $d_i \in \{0, 1\}$. This is a contradiction.

\end{proof}

We arrived at a neat result: if the binary expansion of $\frac{a}{a+b}$ is finite with $k$ digits, then the loop terminates after $k$ steps.

What can we say if the expansion is not finite but instead has a repeating pattern with a prefix and a period (the only other option \footnote{That is because $\frac{a}{a+b} \in \mathbb{Q}$. See below for why.}) ? For starters, we can use a contradiction similar to the earlier one to prove that the loop does not terminate. Consider the infinite binary expansion:

$$
  \frac{a}{n} = \sum_{i = 1}^\infty d_i 2^{-i}
$$

Assume there is a $k$ for which $a_k = 0$. Then by multiplying the expansion with $2^k n$ we get:

$$
  2^k a = \sum_{i = 1}^k n d_i 2^{k-i} + \sum_{i = k+1}^\infty n d_i 2^{k-i} \equiv 0 \mod n
$$

So for some $q \in \mathbb{Z}$ such that $2^k a = n q$ we have

$$
  q = \sum_{i = 1}^k d_i 2^{k-i} + \sum_{i = k+1}^\infty d_i 2^{k-i}
$$

The left side and the first sum on the right both belong to $\mathbb{Z}$ but the second sum does not, which is a contradiction. This means, that $\forall k: a_k > 0$ and the loop does not terminate.

At this point we will do a small digression and prove some theorems about decimal expansion. 

\begin{thm}\label{while_a_decimal_expansion_one}
Given an integer $p > 1$, the series

$$
\sum_{i=1}^\infty \frac{d_i}{p^i}
$$

with $d_i \in \{0, 1, \ldots, p-1\}$ converges to a value $x \in [0, 1]$.
\end{thm}

\begin{proof}

$$
\sum_{i=1}^n \frac{d_i}{p^i} \leq \sum_{i=1}^n \frac{p-1}{p^i} \xrightarrow[n \to \infty]{} 1
$$

so the series is bounded and will converge.
\end{proof}

\begin{thm}\label{while_a_decimal_expansion_one}
For every $x \in [0, 1]$ there exists a decimal expansion with base $p > 1$ such that
$$
x = \sum_{i=1}^\infty \frac{d_i}{p^i}
$$

with $d_i \in \{0, 1, \ldots, p-1\}$.
\end{thm}

\begin{proof}

We divide the interval $[0, 1]$ into $p$ intervals $[\frac{i}{p}, \frac{i+1}{p}]$ with $0 \leq i < p$. Since $[0, 1] = \bigcup_{i = 0}^{p-1} [\frac{i}{p}, \frac{i+1}{p}]$ we know there exists at least one index $i$ with $x \in [\frac{i}{p}, \frac{i+1}{p}]$. We set $d_1 = i$ and subdivide $[\frac{i}{p}, \frac{i+1}{p}]$ into $p$ segments $[\frac{i}{p}, \frac{i+1}{p}] = \bigcup_{j = 0}^{p-1} [\frac{d_1}{p} + \frac{j}{p^2},  \frac{d_1}{p} + \frac{j+1}{p^2}]$. $x$ is in one of these subintervals and we set $d_2$ to be the index of that subinterval and continue in this manner recursively defining all $d_i$. Because of the nested interval property with monotone decreasing length this converges to $x$.

Another way to prove it is like this:

The case where $x=0$ is trivial (just set all $d_i=0$). 

For $x > 0$ we have:

The set $N_1 = \{k \in \mathbb{N}_0: \frac{k}{p} < x\}$ is a set of non-negative integers strictly bounded above by $p$, so it has a largest element and we set $d_1 = \text{max}(N_1)$. Then $x \leq \frac{d_1 + 1}{p}$ (otherwise $d_1 + 1 \in N_1$ and $d_1$ wouldn't be the largest element of $N_1$). We therefore have

$$
\frac{d_1}{p} < x \leq \frac{d_1 + 1}{p}
$$

We continue and look at $N_2 = \{k \in \mathbb{N}_0: \frac{d_1}{p} + \frac{k}{p^2} < x\}$. Again the set $N_2$ is strictly bounded above by $p$ and we set $d_2 = \text{max}(N_2)$. Again we have:

$$
\frac{d_1}{p} + \frac{d_2}{p^2} < x \leq \frac{d_1}{p} + \frac{d_2 + 1}{p^2}
$$

Having defined $d_1, d_2, \ldots d_{n-1}$ we can recursively define $d_n=\text{max}(N_n)$ with 

$$
N_n = \{k \in \mathbb{N}_0: \sum_{i = 1}^{n-1} \frac{d_i}{p^i} + \frac{k}{p^n} < x\}
$$

Again $p \notin N_n$, so the definition is valid and the following inequalities hold:

$$
\sum_{i = 1}^n \frac{d_i}{p^i} < x \leq \sum_{i = 1}^{n-1} \frac{d_i}{p^i} + \frac{d_n + 1}{p^n}
$$

We define $u_n = \sum_{i = 1}^n \frac{d_i}{p^i}$, $v_n = \sum_{i = 1}^{n-1} \frac{d_i}{p^i} + \frac{d_n + 1}{p^n}$ and $w_n = \frac{d_{n+1} + 1}{p^{n+1}}$. $u_n$ is monotone increasing and bounded above, so it converges. For $v_n$ we have

\begin{align*}
& v_n \geq v_{n+1} \\
&\Leftrightarrow \sum_{i = 1}^{n-1} \frac{d_i}{p^i} + \frac{d_n + 1}{p^n} \geq \sum_{i = 1}^n \frac{d_i}{p^i} + \frac{d_{n+1} + 1}{p^{n+1}} \\
&\Leftrightarrow \frac{d_n + 1}{p^n} \geq \frac{d_n}{p^n} + \frac{d_{n+1} + 1}{p^{n+1}} \\
&\Leftrightarrow \frac{1}{p^n} \geq \frac{d_{n+1} + 1}{p^{n+1}} \\
&\Leftrightarrow p \geq d_{n+1} + 1
\end{align*}

which holds by definition of $d_{n+1}$. So $v_n$ is monotone decreasing and bounded below, therefore it converges too. $w_n$ converges to zero and $v_n = u_{n-1} + w_n$ therefore 

$$
lim_{n \to \infty} u_n = lim_{n \to \infty} v_n = x
$$

\end{proof}

\begin{thm}\label{while_a_decimal_expansion_two}
Given is base $p > 1$ and
$$
x = \sum_{i=1}^n \frac{d_i}{p^i}
$$

with $d_i \in \{0, 1, \ldots, p-1\}$ and $d_n \neq 0$. Then there are two base $p$ expansions of $x$.
\end{thm}

\begin{proof}

The first expansion is $x = \sum_{i=1}^\infty \frac{d_i}{p^i}$ with $d_i = 0$ for $i > n$. For the second expansion we define the following series:

$$
y = \sum_{i=1}^{n-1} \frac{d_i}{p^i} + \frac{d_n - 1}{p^n} + \sum_{i=n+1}^\infty \frac{p-1}{p^i}
$$

and prove that $y = x$. Then the two expansions are $0.d_1d_2 \ldots d_n00000 \ldots$ and $0.d_1d_2 \ldots (d_n-1)(p-1)(p-1)(p-1) \ldots$

To prove that $y = x$ we look at

\begin{align*}
\sum_{i=n+1}^\infty \frac{p-1}{p^i} &= \frac{p-1}{p^n} \sum_{i=1}^\infty \frac{1}{p^i} \\
                                    &= \frac{p-1}{p^n} (\sum_{i=0}^\infty \frac{1}{p^i} - 1) \\
                                    &= \frac{p-1}{p^n} (\frac{p}{p-1} - 1) \\
                                    &= \frac{p-1}{p^n} \frac{1}{p-1} \\
                                    &= \frac{1}{p^n}
\end{align*}

So $y$ becomes

$$
y = x - \frac{1}{p^n} + \frac{1}{p^n} = x
$$
\end{proof}

\begin{thm}\label{while_a_decimal_expansion_three}
If we disallow series with infinitely repeated $(p-1)$ tail, any $x \in [0, 1]$ has a unique decimal expansion in base $p$.
\end{thm}

\begin{proof}
Assume two decimal expansions where both agree until index $k-1$ and index $k$ is the first index where they differ.

\begin{align*}
x &= \sum_{i = 1}^{k-1} \frac{d_i}{p^i} + \frac{e_k}{p^k} + \sum_{i = k + 1}^\infty \frac{e_i}{p^i} \\
y &= \sum_{i = 1}^{k-1} \frac{d_i}{p^i} + \frac{f_k}{p^k} + \sum_{i = k + 1}^\infty \frac{f_i}{p^i}
\end{align*}

Without loss of generality assume $e_k < f_k$. 

We have

\begin{align*}
y - x &= \sum_{i = 1}^{k-1} \frac{d_i}{p^i} + \frac{f_k}{p^k} + \sum_{i = k + 1}^\infty \frac{f_i}{p^i} - \sum_{i = 1}^{k-1} \frac{d_i}{p^i} - \frac{e_k}{p^k} - \sum_{i = k + 1}^\infty \frac{e_i}{p^i} \\
      &= \frac{f_k - e_k}{p^k} + \sum_{i = k + 1}^\infty \frac{f_i}{p^i} - \sum_{i = k + 1}^\infty \frac{e_i}{p^i} \\
      &= \frac{f_k - e_k}{p^k} + \frac{1}{p^k} (\sum_{i = 1}^\infty \frac{f_{k+i}}{p^i} - \sum_{i = 1}^\infty \frac{e_{k+i}}{p^i})
\end{align*}

We denote $u = \sum_{i = 1}^\infty \frac{f_{k+i}}{p^i}$ and $v = \sum_{i = 1}^\infty \frac{f_{k+i}}{p^i}$. Since we disallowed repeated $(p-1)$ tail, we know that $0 \leq u < 1$ and $0 \leq v < 1$, so $-1 < u - v < 1$. It follows that

$$
0 \leq \frac{f_k - e_k - 1}{p^k} < y - x < \frac{f_k - e_k + 1}{p^k}
$$

and $x \neq y$.
\end{proof}

\begin{thm}\label{while_a_decimal_expansion_four}
$x \in [0, 1] \cap \mathbb{Q}$ if and only if its decimal expansion in base $p > 1$ is either finite or has a prefix (of length zero or more) and an infinitely repeating non-zero length pattern tail.
\end{thm}

\begin{proof} \

$(\Rightarrow)$:

$x \in [0, 1] \cap \mathbb{Q}$, so there exist $m, n \in \mathbb{N}$ with $m < n$ and $x = \frac{m}{n}$. We basically do the long division and present an expansion that will have a repeating tail (if it isn't finite). Let $k \in \mathbb{N}$ be the smallest integer such that $m p^k \geq n$ and we do division:

$$
m p^k = n q + r
$$

with $0 \leq r < n$. Because $k$ is the smallest integer with $m p^k \geq n$ we have $n p > m p^k$ (otherwise $k-1$ would be a smaller integer satisfying the same). That means $n p > n q + r$ and thus $p > \frac{n p - r}{n} > q$. This gives us $k-1$ zeros and the first non-zero digit in the expansion, namely $q$:

\begin{align*}
\frac{m}{n} &= \frac{1}{p^k} \frac{m p^k}{n} \\
            &= \frac{1}{p^k} \frac{ n q + r}{n} \\
            &= \frac{q}{p^k} + \frac{r}{n}
\end{align*}

We repeat this process with $\frac{r}{n}$. There are only $n$ possible remainders, so if it doesn't end with a remainder of zero it must eventually get a previously seen remainder and so the expansion will repeat itself. This creates an expansion with an infinitely repeating non-zero length pattern tail. Since it isn't finite, we can disallow repeating $(p-1)$ and from the expansion uniqueness theorem we have proved the $(\Rightarrow)$ direction.

$(\Leftarrow)$:

This direction is easy. If it is a finite sum, then it is rational since all the parts are rational. If it is infinite repeating we can eliminate the non-repeating prefix since it is finite and rational and shift the rest. So we can concentrate on a repeating series with a period $k-1$:

\begin{align*}
x &= \sum_{i = 0}^\infty (\frac{1}{p^{k i}} \sum_{j = 1}^{k-1} \frac{d_j}{p^j}) \\
  &= (\sum_{j = 1}^{k-1} \frac{d_j}{p^j}) \sum_{i = 0}^\infty \frac{1}{p^{k i}} \\
  &= (\sum_{j = 1}^{k-1} \frac{d_j}{p^j}) (1 + \sum_{i = 1}^\infty \frac{1}{p^{k i}}) \\
  &= (\sum_{j = 1}^{k-1} \frac{d_j}{p^j}) (1 + \sum_{i = 1}^\infty (\frac{1}{p^{k}})^i) \\
  &= (\sum_{j = 1}^{k-1} \frac{d_j}{p^j}) (1 + \frac{p^k}{p^k -1})
\end{align*}

which is a rational expression.
\end{proof}

We return to our problem. We now know the expansion of $\frac{a}{a+b}$ is repeating a period if it doesn't terminate. We will show that the loop also repeats a period of the same length.

\todo{Finish up.}
