\documentclass[justified, openany]{tufte-book}

\usepackage[utf8]{inputenc}
\usepackage[english]{babel}
\usepackage{blindtext}
\usepackage{todonotes}

\setcounter{secnumdepth}{1}
\setcounter{tocdepth}{1}

% turn on numbering for parts and chapters

\usepackage{amsthm, amsmath, amssymb}
\usepackage{setspace, graphicx, enumerate}

\usepackage{stmaryrd}

% For nicely typeset tabular material
\usepackage{booktabs}

\usepackage{listings}
\lstloadlanguages{Haskell}

\usepackage{permute}

\newcommand{\monthyear}{%
  \ifcase\month\or January\or February\or March\or April\or May\or June\or
  July\or August\or September\or October\or November\or
  December\fi\space\number\year
}

% For graphics / images
\usepackage{graphicx}
\usepackage{fancyvrb}
\fvset{fontsize=\normalsize}
\usepackage{xspace}

\usepackage{pgf, tikz}
\usetikzlibrary{arrows,automata,decorations.pathmorphing,backgrounds,positioning,fit,petri,shapes.geometric,calc}

\usepackage{units}

\usepackage{bibentry}

\usepackage{tcolorbox}

\usepackage{enumitem}

\usepackage{bm}

\usepackage{mmacells}

\usepackage{oplotsymbl}

\usepackage{pdfpages}

\usepackage[nottoc]{tocbibind}

\theoremstyle{plain}% default 
\newtheorem{thm}{Theorem}[chapter] 
\newtheorem{lem}[thm]{Lemma} 
\newtheorem{prop}[thm]{Proposition} 
\newtheorem*{cor}{Corollary} 

\theoremstyle{definition} 
\newtheorem{defn}[thm]{Definition}
\newtheorem{conj}[thm]{Conjecture}
\newtheorem{exmp}[thm]{Example}
\newtheorem{exer}[thm]{Exercise}

\newtheorem{proofpart}{Proof Part}[thm]

\theoremstyle{remark} 
\newtheorem*{rem}{Remark} 
\newtheorem*{note}{Note} 
\newtheorem{case}{Case}
\newtheorem{thminv}{Invariant}[chapter] 


\newtcolorbox{problem}{title={Problem}}

\newenvironment{dedication}
    {\vspace{6ex}\begin{quotation}\begin{center}\begin{em}\begin{large}}
    {\par\end{large}\end{em}\end{center}\end{quotation}}

\mmaDefineMathReplacement[≤]{<=}{\leq}
\mmaDefineMathReplacement[≥]{>=}{\geq}
\mmaDefineMathReplacement[≠]{!=}{\neq}
\mmaDefineMathReplacement[→]{->}{\to}[2]
\mmaDefineMathReplacement[⧴]{:>}{:\hspace{-.2em}\to}[2]
\mmaDefineMathReplacement{∉}{\notin}
\mmaDefineMathReplacement{∞}{\infty}
\mmaDefineMathReplacement{𝕕}{\mathbbm{d}}


\mmaSet{
  morefv={gobble=2},
  linklocaluri=mma/symbol/definition:#1,
  morecellgraphics={yoffset=1.9ex}
}



\title{Math notes - While a}
\author{Uwe Hoffmann}
\hypersetup{colorlinks, pdftitle={Math notes - While a}}

\newtheorem{thminv}{Invariant}[chapter] 

\begin{document}

\setcounter{chapter}{1}
\section*{While a}

\newthought{Loop invariants} is the topic of the problem \footnote{Problem 4 on page 9 from \bibentry{engel2013problem}} in this note.\index{loop invariants}


\vspace{10 mm}
\begin{problem}
We start with the state $(a, b)$ where $a$, $b$ are positive integers. To this initial state we apply the following algorithm:

\begin{lstlisting}[language=Python, basicstyle=\small, label={lst:while_loop}, frame=trBL]
while a > 0:
   if a < b:
      (a,b) = (2a, b - a)
   else:
      (a,b) = (a - b, 2b) 
\end{lstlisting}

For which starting positions does the algorithm stop? In how many steps does it stop,
if it stops? What can you tell about periods and tails?

\end{problem}

We start with $a > 0$ and $b > 0$. We adopt the following notation: $a_i$, $b_i$ are the values after $i \in \mathbb{N}_{\geq 0}$ times through the loop. Before the first time through the loop $a_0 = a$, $b_0 = b$. Let $n = a + b$.

Let's collect some invariants. We will prove all of them by induction on $i \in \mathbb{N}_{\geq 0}$.

\begin{thminv}\label{p1}
$$
  \forall i \geq 0: a_i + b_i = n
$$
\end{thminv}

\begin{proof}
Base case $a_0 + b_0 = a + b = n$ holds by definition of $n$ and $(a_0, b_0)$.
Assume $a_i + b_i = n$. For $a_{i+1} + b_{i+1}$ we have two cases:

Case $a_i < b_i$: Here we have $a_{i+1} = 2 a_i$ and $b_{i+1} = b_i - a_i$. So

$$
a_{i+1} + b_{i+1} = 2 a_i + b_i - a_i = a_i + b_i = n
$$

Case $a_i \geq b_i$: In this case we have  $a_{i+1} = a_i - b_i$ and $b_{i+1} = 2 b_i$. It follows

$$
a_{i+1} + b_{i+1} = a_i - b_i + 2 b_i = a_i + b_i = n
$$

\end{proof}

\begin{thminv}\label{p2}
$$
  \forall i \geq 0: b_i > 0
$$
\end{thminv}

\begin{proof}
This follows almost immediately from definitions \footnote{
Base case $b_0 = b > 0$ holds by definition of $b$. Assume $b_i > 0$. Again we have two cases. If $a_i < b_i$ then $b_{i+1} = b_i - a_i > 0$. 
If $a_i \geq b_i$ then $b_{i+1} = 2 b_i > 0$.
}.

\end{proof}

\begin{thminv}\label{p3}
$$
  \forall i \geq 0: a_i \geq 0
$$
\end{thminv}

\begin{proof}
This also follows from definitions \footnote{
Base case $a_0 = a > 0$ holds by definition of $a$. Assume $a_i \geq 0$. Again we have two cases. If $a_i < b_i$ then $a_{i+1} = 2 a_i \geq 0$. 
If $a_i \geq b_i$ then $a_{i+1} = a_i - b_i \geq 0$.
}.

\end{proof}


\begin{thminv}\label{p4}
$$
  \forall i \geq 0: a_i \equiv 2^i a \mod n
$$
\end{thminv}

\begin{proof}
Base case $a_0 = a = 2^0 a$ trivially holds.
Assume $a_i \equiv 2^i a \mod n$. For $a_{i+1}$ we have two cases:

Case $a_i < b_i$: Here we have $a_{i+1} = 2 a_i$. So

\begin{align*}
  a_{i+1} &= 2 a_i \\
  &\equiv 2 \cdot 2^i a \mod n \\
  &\equiv 2^{i+1} a \mod n
\end{align*}

Case $a_i \geq b_i$: In this case we have  $a_{i+1} = a_i - b_i$. It follows

\begin{align*}
  a_{i+1} &= a_i - b_i \\
  &\equiv a_i + n - b_i \mod n \\
  &\equiv a_i + a_i + b_i - b_i \mod n \\
  &\equiv 2 a_i \mod n \\
  &\equiv 2 \cdot 2^i a \mod n \\
  &\equiv 2^{i+1} a \mod n  
\end{align*}

\end{proof}

We will use these 4 invariants ($a_i \geq 0$, $b_i > 0$, $a_i + b_i = n$ and $a_i \equiv 2^i a \mod n$) to determine for which initial values $a$ and $b$ the loop terminates. To do so we consider $\frac{a}{n}$. Because $0 < a < n$ we know that $\frac{a}{n} \in (0, 1)$. We look at the expansion of $\frac{a}{n}$ in base $2$.

\begin{thm}
If the expansion of $\frac{a}{n}$ is finite with $k$ digits $d_i \in \{0, 1\}$

$$
  \frac{a}{n} = \sum_{i = 1}^k d_i 2^{-i}
$$

then $a_k = 0$ and the loop terminates after $k$ steps.
\end{thm}

\begin{proof}

From 

$$
  \frac{a}{n} = \sum_{i = 1}^k d_i 2^{-i}
$$

we get by multiplying both sides with $2^k n$:

$$
  2^k a = \sum_{i = 1}^k n d_i 2^{k-i} \equiv 0 \mod n
$$

Together with invariant \ref{p4} we get

$$
  a_k \equiv 2^k a \equiv 0 \mod n
$$

and because $a_k \geq 0$, $b_k > 0$, $a_k + b_k = n$ we know that $0 \leq a_k < n$, so it must be that $a_k=0$ and the loop terminates after at most $k$ steps. To show that the loop terminates after exactly $k$ steps, we need to show that $a_j > 0$ for $0 \leq j < k$. We will do this by finding a contradiction. Assume there exists a $j < k$ such that $a_j = 0$. Then it also holds that $2^j a \equiv 0 \mod n$.

From 

$$
  \frac{a}{n} = \sum_{i = 1}^k d_i 2^{-i}
$$

we get by multiplying both sides with $2^j n$:

$$
  2^j a = \sum_{i = 1}^k n d_i 2^{j-i} = \sum_{i = 1}^j n d_i 2^{j-i} + \sum_{i = j+1}^k n d_i 2^{j-i} \equiv 0 \mod n
$$

$2^j a \equiv 0 \mod n$, so $2^j a = n q$ for some $q \in \mathbb{Z}$. Then

$$
  q = \sum_{i = 1}^j d_i 2^{j-i} + \sum_{i = j+1}^k d_i 2^{j-i}
$$

We have $q \in \mathbb{Z}$, $\sum_{i = 1}^j d_i 2^{j-i} \in \mathbb{Z}$, but $\sum_{i = j+1}^k d_i 2^{j-i} \notin \mathbb{Z}$, because $d_i \in \{0, 1\}$. This is a contradiction.

\end{proof}

We arrived at a neat result: if the binary expansion of $\frac{a}{a+b}$ is finite with $k$ digits, then the loop terminates after $k$ steps.

What can we say if the expansion is not finite but instead has a repeating pattern with a prefix and a period (the only other option \footnote{That is because $\frac{a}{a+b} \in \mathbb{Q}$.}) ? For starters, we can use a contradiction similar to the earlier one to prove that the loop does not terminate. Consider the infinite binary expansion:

$$
  \frac{a}{n} = \sum_{i = 1}^\infty d_i 2^{-i}
$$

Assume there is a $k$ for which $a_k = 0$. Then by multiplying the expansion with $2^k n$ we get:

$$
  2^k a = \sum_{i = 1}^k n d_i 2^{k-i} + \sum_{i = k+1}^\infty n d_i 2^{k-i} \equiv 0 \mod n
$$

So for some $q \in \mathbb{Z}$ such that $2^k a = n q$ we have

$$
  q = \sum_{i = 1}^k d_i 2^{k-i} + \sum_{i = k+1}^\infty d_i 2^{k-i}
$$

The left side and the first sum on the right both belong to $\mathbb{Z}$ but the second sum does not, which is a contradiction. This means, that $\forall k: a_k > 0$ and the loop does not terminate.

\bibliographystyle{plainnat}
\bibliography{../common/math}

\end{document}

