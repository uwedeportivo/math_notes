\documentclass[justified, openany]{tufte-book}

\usepackage[utf8]{inputenc}
\usepackage[english]{babel}
\usepackage{blindtext}
\usepackage{todonotes}

\setcounter{secnumdepth}{1}
\setcounter{tocdepth}{1}

% turn on numbering for parts and chapters

\usepackage{amsthm, amsmath, amssymb}
\usepackage{setspace, graphicx, enumerate}

\usepackage{stmaryrd}

% For nicely typeset tabular material
\usepackage{booktabs}

\usepackage{listings}
\lstloadlanguages{Haskell}

\usepackage{permute}

\newcommand{\monthyear}{%
  \ifcase\month\or January\or February\or March\or April\or May\or June\or
  July\or August\or September\or October\or November\or
  December\fi\space\number\year
}

% For graphics / images
\usepackage{graphicx}
\usepackage{fancyvrb}
\fvset{fontsize=\normalsize}
\usepackage{xspace}

\usepackage{pgf, tikz}
\usetikzlibrary{arrows,automata,decorations.pathmorphing,backgrounds,positioning,fit,petri,shapes.geometric,calc}

\usepackage{units}

\usepackage{bibentry}

\usepackage{tcolorbox}

\usepackage{enumitem}

\usepackage{bm}

\usepackage{mmacells}

\usepackage{oplotsymbl}

\usepackage{pdfpages}

\usepackage[nottoc]{tocbibind}

\theoremstyle{plain}% default 
\newtheorem{thm}{Theorem}[chapter] 
\newtheorem{lem}[thm]{Lemma} 
\newtheorem{prop}[thm]{Proposition} 
\newtheorem*{cor}{Corollary} 

\theoremstyle{definition} 
\newtheorem{defn}[thm]{Definition}
\newtheorem{conj}[thm]{Conjecture}
\newtheorem{exmp}[thm]{Example}
\newtheorem{exer}[thm]{Exercise}

\newtheorem{proofpart}{Proof Part}[thm]

\theoremstyle{remark} 
\newtheorem*{rem}{Remark} 
\newtheorem*{note}{Note} 
\newtheorem{case}{Case}
\newtheorem{thminv}{Invariant}[chapter] 


\newtcolorbox{problem}{title={Problem}}

\newenvironment{dedication}
    {\vspace{6ex}\begin{quotation}\begin{center}\begin{em}\begin{large}}
    {\par\end{large}\end{em}\end{center}\end{quotation}}

\mmaDefineMathReplacement[≤]{<=}{\leq}
\mmaDefineMathReplacement[≥]{>=}{\geq}
\mmaDefineMathReplacement[≠]{!=}{\neq}
\mmaDefineMathReplacement[→]{->}{\to}[2]
\mmaDefineMathReplacement[⧴]{:>}{:\hspace{-.2em}\to}[2]
\mmaDefineMathReplacement{∉}{\notin}
\mmaDefineMathReplacement{∞}{\infty}
\mmaDefineMathReplacement{𝕕}{\mathbbm{d}}


\mmaSet{
  morefv={gobble=2},
  linklocaluri=mma/symbol/definition:#1,
  morecellgraphics={yoffset=1.9ex}
}



\title{Math notes - Paying a dollar}
\author{Uwe Hoffmann}
\hypersetup{colorlinks, pdftitle={Math notes - Paying a dollar}}

\begin{document}

\setcounter{chapter}{1}
\section*{Paying a dollar}

\vspace{10 mm}
\begin{problem}
In how many combinations of half-dollars, quarters, dimes, nickels and pennies  can you pay out one dollar ? You can assume you have enough coins for any combination and any coins of one denomination are indistinguishable.
\end{problem}

We will look at tuples $(h, q, d, n, p)$ where $h$ is the number of half-dollars, $q$ the number of quarters, $d$ the number of dimes, $n$ the number of nickels and $p$ the number of pennies. We want to consider all tuples $(h, q, d, n, p)$ such that:

\begin{equation*}
50 h + 25 q + 10 d + 5 n + p = 100
\end{equation*} 

We want to determine how many such tuples exist. Let's establish what the possible values for $h$, $q$, $d$, $n$ and $p$ can be:

\begin{center}
  \begin{tabular}{cll}
       Denomination & Possible Values & Values range size\\[5pt]
       h & 0, 1, 2 & 3\\
       q & 0, 1, 2, 3, 4 & 5\\
       d & 0, \ldots, 10 & 11\\
       n & 0, \ldots, 20 & 21\\
       p & 0, \ldots, 100 & 101
    \end{tabular}
\end{center}    

Clearly $h$, $q$, $d$, $n$ and $p$ cannot take values outside of the ones listed because it would violate the requirement: $50 h + 25 q + 10 d + 5 n + p = 100$.

The multiplicity principle tells us there are $3 * 5 * 11 * 21 * 101 =  349965$ distinct tuples from the possible values.

But not all of them fulfill $50 h + 25 q + 10 d + 5 n + p = 100$. To figure out how many of them do let's introduce the following notation:

\begin{equation*}
\#(h, q, d, n, p)_x = | \{(h, q, d, n, p) : 50 h + 25 q + 10 d + 5 n + p = x\} |
\end{equation*} 

so $\#(h, q, d, n, p)_x$ is the number of tuples of half-dollars, quarters, dimes, nickels and pennies that add up to $x$. We are looking for $\#(h, q, d, n, p)_{100}$. 

We also use $\#(q, d, n, p)_x$ for tuples of quarters, dimes, nickels and pennies that add up to $x$, $\#(d, n, p)_x$ for dimes, nickels and pennies that add up to $x$ etc.

The half-dollar denomination has the smallest values range size so it's probably in our favor to start with it and break down the problem into smaller problems from there. It is clear that

\begin{equation*}
\#(h, q, d, n, p)_{100} = \#(q, d, n, p)_{100} + \#(q, d, n, p)_{50} + \#(q, d, n, p)_0
\end{equation*} 

because $\#(q, d, n, p)_{100}$ comes from $h$ taking value $0$, $\#(q, d, n, p)_{50}$ from $h$ taking value $1$ and $ \#(q, d, n, p)_0$ from $h$ taking value $2$. Continuing down this path and breaking down the $q$ cases:

\begin{equation*}
    \begin{split}
      \#(q, d, n, p)_{100} & = \#(d, n, p)_{100} + \#(d, n, p)_{75} +\\
      &  \#(d, n, p)_{50} + \#(d, n, p)_{25} + \#(d, n, p)_0\\
      \#(q, d, n, p)_{50} & = \#(d, n, p)_{50} + \#(d, n, p)_{25} + \#(d, n, p)_0\\
      \#(q, d, n, p)_{0} & = \#(d, n, p)_0
    \end{split}
\end{equation*} 

This is getting tedious though. Maybe we can find a closed-form formula for $\#(d, n, p)_{x}$. Let's try to find one for $\#(n, p)_{x}$ first.

\begin{lem}\label{np}
If $x = 5y$ then
\begin{equation*}
\#(n, p)_{x} = y + 1
\end{equation*}  
\end{lem}

\begin{proof}
 $n$ can take values in range $0, \ldots, y$ and for each value of $n$ there is only one possible value of $p = x - 5n$.
\end{proof}

\begin{lem}\label{dnp}
If $x$ is a multiple of $5$ then
\begin{equation*}
\#(d, n, p)_{x} = 
   \begin{cases}
       (y + 1)^2 & \text{if $x = 10 y$}\\
       (y + 1)(y + 2) & \text{if $x = 10y + 5$}
    \end{cases}   
\end{equation*}  
\end{lem}

\begin{proof}
 
Let's deal with the case $x = 10y$ first.

\begin{equation*}
    \begin{split}
      \#(d, n, p)_{x} & = \sum_{k = 0}^{y} \#(n, p)_{x - 10k}\\
                               & =  \sum_{k = 0}^{y} \#(n, p)_{10 (y - k)} \\
                               & =  \sum_{k = 0}^{y} \#(n, p)_{10 k} \\
                               & =  \sum_{k = 0}^{y} (2 k + 1) \\
                               & = (y + 1)^2
    \end{split}
\end{equation*}  

The case $x = 10y + 5$ is established in a similar fashion.
\end{proof}

We can now use the two lemmas to compute the number of combinations:

\begin{equation*}
    \begin{split}
     \#(h, q, d, n, p)_{100} & =\#(d, n, p)_{100} +  \#(d, n, p)_{75} +\\
            & 2 \#(d, n, p)_{50} + 2 \#(d, n, p)_{25} + 3 \#(d, n, p)_0\\
                               & =  11^2 +  8 * 9 + 2 * 6^2 + 2 * 3 * 4 + 3\\
                               & = 292
    \end{split}
\end{equation*}  

\end{document}
